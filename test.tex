\documentclass{article}

\usepackage[utf8]{inputenc} % 设置输入编码
\usepackage{amsmath} % 数学包
\usepackage{graphicx} % 图形包
\usepackage{hyperref} % 超链接包

\title{A Test Document}
\author{Ruilong Chen}
\date{\today}
\lstset{
  language=Python,              % Set language to Python
  basicstyle=\ttfamily,         % Use monospaced font for code
  numbers=left,                 % Line numbers on the left
  frame=single,                 % Frame around the code block
  backgroundcolor=\color{lightgray}, % Background color
  keywordstyle=\color{blue},    % Color for keywords
  commentstyle=\color{gray},    % Color for comments
  stringstyle=\color{green},    % Color for strings
  breaklines=true,              % Enable automatic line breaking
  breakindent=0pt,              % No indent after a line break
  postbreak=\mbox{\textcolor{red}{$\hookrightarrow$}\space}, % Symbol at line break
  escapeinside={(*@}{@*)},      % Allows you to escape to LaTeX
}


\begin{document}

\maketitle

\section{Introduction}

This is a simple test document to demonstrate the basic structure of a LaTeX file. LaTeX is a powerful typesetting system commonly used for academic and scientific documents.

\section{Mathematics}

Here is an example of a mathematical equation:

\begin{equation}
E = mc^2
\end{equation}

This equation expresses the equivalence of mass and energy.

\section{Figures}

You can also include figures in your document. Below is an example:
 
how about that

\begin{figure}[h]
    \centering
    \includegraphics[width=0.5\textwidth]{example-image} % 使用示例图像
    \caption{An example image.}
    \label{fig:example}
\end{figure}

\section{Conclusion}

In conclusion, LaTeX is a versatile tool for creating well-structured documents. You can include various elements such as equations, figures, and references.

\end{document}