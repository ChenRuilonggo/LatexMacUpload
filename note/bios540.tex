\documentclass{article}
\usepackage{lmodern}  % 或者 cm-super
\usepackage[T1]{fontenc}  % 让 LaTeX 使用 T1 字体编码
\usepackage[utf8]{inputenc}  % 确保使用 UTF-8 编码
\usepackage{amsmath} % 数学包
\usepackage{graphicx} % 图形包
%\usepackage{ctex}
\usepackage[colorlinks=true, linkcolor=blue, urlcolor=blue, citecolor=blue]{hyperref}
% 超链接包
\usepackage[a4paper, margin=1in]{geometry}
\usepackage{listings} %可以插入代码
\usepackage{xcolor} %可以定义颜色
\setlength{\parindent}{0pt}  % 取消缩进
\setlength{\parskip}{1em}    % 增加段落之间的空行

\begin{document}

\section{Sequence Motif Analysis}
\subsection{What is sequence motif?}
\textbf{Transcription regulation in prokarynote VS. Transcription regulation in Eukaryote}


•	Prokaryotes: Simple, rapid regulation, mainly at the level of transcription initiation, with a focus on immediate environmental response.

•	Eukaryotes: Complex, multilayered regulation involving chromatin modifications, transcription factors, and various RNA processing events. Transcription regulation is slower but allows for more fine-tuned control of gene expression.

\textbf{Binding Motif}

A binding motif refers to a specific sequence or structure in DNA, RNA, or proteins that is recognized and bound by a particular molecule, usually a protein such as a transcription factor. In the context of DNA-binding proteins, the motif typically refers to the sequence of nucleotides or the structural features that the protein binds to regulate gene expression or perform other functions.

\begin{itemize}
    \item DNA-binding Motif
    \item Rna-binding Motif
    \item protein-binding motif
\end{itemize}

\subsection{Why is motif important?}
\subsection{Motif Analysis}
\subsection{Mixture model}






\end{document}