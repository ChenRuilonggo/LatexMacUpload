\documentclass{article}
\usepackage{lmodern}  % 或者 cm-super
\usepackage[T1]{fontenc}  % 让 LaTeX 使用 T1 字体编码
\usepackage[utf8]{inputenc}  % 确保使用 UTF-8 编码
\usepackage{amsmath} % 数学包
\newtheorem{definition}{Definition}
\usepackage{graphicx} % 图形包
%\usepackage{ctex}
\usepackage[colorlinks=true, linkcolor=blue, urlcolor=blue, citecolor=blue]{hyperref}
% 超链接包
\usepackage[a4paper, margin=1in]{geometry}
\usepackage{listings} %可以插入代码
\usepackage{xcolor} %可以定义颜色
\setlength{\parindent}{0pt}  % 取消缩进
\setlength{\parskip}{0.5em}    % 增加段落之间的空行


\begin{document}

\section{Experimental Studies and Cohort Studies Review}
\subsection{Experimemtal Vs Obsercational}
\begin{definition}
Experimental

\textbf{A study in which the conditions ara under direct control of the investigator.(assign individuals to exposed or unexposed)}

\end{definition}
\begin{itemize}
    \item RCT
    \item cinical trial
\end{itemize}

\begin{definition}
    Observational
    
    \textbf{The investigator does not determine the assignment of exposure but rather passively observe events as they occur}
    \end{definition}
    
    
    
    \begin{itemize}
        \item Cohort
        \item Case-control
        \item cross-sectional
    \end{itemize}


\subsection{Clinical Trial}  
Catorgaries

\begin{itemize}
    \item with ramdomization
    \item without randomazation(Simple randomazation, stratified randomazation,cross-over , factorial design)
\end{itemize}

\textbf{Purpose of randomazation}

\begin{itemize}
    \item Achieve equality of baseline characteristics, allow for a fair comparison
    \item Balance all factors (known and unknown) that may bias our final results
    \item If we find some differences between groups, this difference can only be attributable to the exposure (and not any extraneous factors)
\end{itemize}


\end{document}